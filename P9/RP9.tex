\documentclass{article}

\usepackage[spanish]{babel}
\usepackage[numbers,sort&compress]{natbib}
\usepackage{graphicx}
\usepackage{url}
\usepackage{amsmath}
\usepackage{hyperref}
\usepackage{listings}
\usepackage[top=30mm, bottom=40mm, left=15mm, right=15mm]{geometry}
\usepackage{color}
\usepackage{subfig}
\usepackage{float}
 
\definecolor{codeblue}{RGB}{0,128,255}
\definecolor{codegray}{rgb}{0.5,0.5,0.5}
\definecolor{codepurple}{rgb}{0.58,0,0.82}
\definecolor{pink}{RGB}{255,26,117}
\definecolor{backcolour}{rgb}{1,1,1}
 
\lstdefinestyle{mystyle}{
    backgroundcolor=\color{backcolour},
    commentstyle=\color{codeblue},
    keywordstyle=\color{pink},
    numberstyle=\tiny\color{codeblue},
    stringstyle=\color{codepurple},
    basicstyle=\footnotesize,
    breakatwhitespace=false,         
    breaklines=true,                 
    captionpos=b,                    
    keepspaces=true,                 
    numbers=left,                    
    numbersep=7pt,                  
    showspaces=false,                
    showstringspaces=false,
    showtabs=false,                  
    tabsize=2
}

\lstset{style=mystyle}

\setlength{\parskip}{2mm}
\setlength{\parindent}{0pt}

\author{Edson Raúl Cepeda Márquez}
\title{Interacciones entre partículas}
\date{\today}

\begin{document}

\maketitle

\section{Introducción}
El objetivo principal de esta práctica es analizar el comportamiento de las velocidades de un grupo de partículas cuando existen interacciones entre ellas.
Se estudia el efecto en la velocidad con la que se mueven las partículas cuando la magnitud de su carga influye en sus fuerzas de atracción y repulsión y se le agrega masa a las partículas para estudiar el efecto de la fuerza gravitacional sumada a la fuerza de atracción y repulsión de las cargas.
Se hace uso del código escrito en el lenguaje de programación R \cite{r} así como el material de apoyo disponible en la página \cite{satu} de la Dra. Elisa Schaeffer.

\section{Desarrollo}
Se selecciona las partes más importantes de código y se procede a modificarlas.
El primer paso es agregar masa a las partículas, para esto se localiza la sección del código donde se especifican las propiedades de las partículas.


Estas se contienen en un dataframe al se le agrega una nueva variable \texttt{m} y se le dan valores aleatorios en un rango de cero a uno.
\lstinputlisting[language=R,firstline=2, lastline=2]{P9.R}
Seguido se localiza la sección del código que contiene el cálculo de la fuerza, esto está en la función \texttt{fuerza} y se modifica para que se realiza el calculo correspondiente a la atracción gravitatoria que produce la masa sobre las partículas.


Se extraen los valores de la masa de las partículas y se declaran variables para su uso futuro.


También se calcula la dirección hacia donde las partículas deberán moverse cuando exista una fuerza de atracción, esto se hace comprobando cual de las cargas es menor con respecto a la otra y apartir del resultado se establece la dirección, se guarda en la nueva variable \texttt{dirm}.


Se le agrega una nueva regla que indica que la fuerza de atracción producida por la masa decrece con el cuadrado de su distancia entre ellas y sera igual al producto de las masas entre partículas, esto se hace en la nueva variable \texttt{factorm}.
Seguido se suman las fuerzas calculadas con las cargas y las fuerzas calculadas con la masa, se obtienen dos nuevas fuerzas totales y se agregan a las nuevas variables \texttt{fxt} y \texttt{fyt}.

\lstinputlisting[language=R,firstline=24, lastline=51]{P9.R}

Para obtener las fuerzas solo con una de las dos cantidades, masa o carga, se separa en dos funciones nuevas estas son \texttt{fuerzam} y \texttt{fuerzac}, a las cuales se les retira la parte del código que no corresponde al cálculo de la fuerza de esa cantidad.
\lstinputlisting[language=R,firstline=54, lastline=54]{P9.R}
\lstinputlisting[language=R,firstline=81, lastline=81]{P9.R}

A continuación se ejecuta en paralelo el calculo de todas las fuerzas y se guardan en distintos vectores para su graficación posterior.

Esto se hace 100 veces, lo cual se refiere a que las partículas dan 100 pasos en la simulación.
\lstinputlisting[language=R,firstline=114, lastline=130]{P9.R}

Como se pretende comparar como se comporta la velocidad cuando se introduce la masa y también la carga, se realiza una gráfica en la cual se puede visualizar los tres resultados al mismo tiempo.

\section{Conclusiones y resultados}
En la figura \ref{g1} se observa como se comporta la velocidad con la masa, la carga y con ambas.
\begin{figure}[H]
\centering
\includegraphics[width = 180mm]{g1.png}
\caption{Gráfica de las velocidades de las partículas con masa, carga y ambas}
\label{g1}
\end{figure}

En conclusión, cuando se agrega masa a las partículas el rango de velocidad con la que se atraen es mayor puesto que ahora se tienen dos fuerzas influyendo sobre las partículas.
A su vez la velocidad con la que la masa afecta a las partículas incrementa cuando estan más cercas , esto se traduce a un efecto aglutinante por cada paso y se puede observar conforme los pasos son cercanos a 100 los valores de la velocidad con la masa suben rápidamente.
Como resultado se obtiene el mismo efecto aglutinante de la atracción gravitatoria sumado a las fuerzas de atracción y repulsión de las cargas.




\begin{thebibliography}{9}

\bibitem{r} 
R:  R Project, 2019
\\\texttt{https://www.r-project.org/}

\bibitem{satu} 
Satu Elisa Schaeffer: Práctica 9: interacciones entre partículas, 2019
\\\texttt{https://elisa.dyndns-web.com/teaching/comp/par/p9.html}



\end{thebibliography}










\end{document}
